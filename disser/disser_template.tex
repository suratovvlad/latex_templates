% Формат бумаги: А4.
\documentclass[a4paper]{report}

% UTF-8 encoding
\usepackage[utf8]{inputenc}
% Russian
\usepackage[english,russian]{babel}

\usepackage{amsmath,amsfonts,amssymb,cite,url,verbatim,graphicx,wrapfig,physics,seqsplit}

% Подрисунки
\usepackage{subfigure}
\usepackage[subfigure]{tocloft}

\usepackage{booktabs}
\graphicspath{{images/}}

% Поля: верхнее – 2 см, нижнее – 2 см, левое – 3 см, правое – 1.5 см.
%\usepackage[showframe]{geometry} % showframe - показать рамки
\usepackage{geometry}
\geometry{
	left = 3cm,
	top = 2cm,
	right = 1.5cm,
	bottom = 2cm
}

% Кегль:
%% текст в таблице, подписи к рисункам, таблицам – 12 пт,
\renewcommand{\small}{\fontsize{12}{12}\selectfont}
%% основной текст – 14 пт,
\renewcommand{\normalsize}{\fontsize{14}{14}\selectfont}
%% названия параграфов – 16 пт,
\renewcommand{\large}{\fontsize{16}{16}\selectfont}
%% названия глав – 18 пт,
\renewcommand{\Large}{\fontsize{18}{18}\selectfont}
\renewcommand{\Huge}{\fontsize{20}{20}\selectfont}


\usepackage{sectsty}
\sectionfont{\Large}
\subsectionfont{\large}
\paragraphfont{\normalsize}
\usepackage[font=small]{caption}

% Межстрочный интервал: 1.5 строки.
\linespread{1.5}

% Абзацный отступ. Первая строка каждого абзаца должна иметь абзацный отступ 1.25 см.
\usepackage{indentfirst}
\setlength{\parindent}{1.25cm}

% Выравнивание основного текста по ширине поля.
\usepackage{ragged2e}
\justifying

% Отключение расстановки переносов + нормальная ширина строки
\pretolerance=1000
\hyphenpenalty=10000
\tolerance=2000 
\emergencystretch=25pt

%%% здесь лучше ничего не трогать
% Глубина: 1=глава-раздел
\setcounter{tocdepth}{1}

% Выключение нумерации раздела
%\renewcommand{\thesection}{}
% Выключение нумерации подраздела
%\renewcommand{\thesubsection}{}

% Выравниваение параграфов (разделов) по центру
\allsectionsfont{\centering}

\usepackage{titlesec}
\usepackage{titletoc}% http://ctan.org/pkg/titletoc
% Какая-то команда
%\newcommand{\sectionbreak}{\clearpage}

%\cftsetindents{chapter}{0em}{0em}
%\cftsetindents{subsection}{2em}{0em}
% Какие-то отступы
%\cftsetindents{section}{0em}{0em}
%\cftsetindents{subsection}{2em}{0em}


% Настройка отображения главы в тексте
\titleformat{\chapter}[display]
	{\filcenter}
	{\bfseries \Large{\chaptertitlename~\thechapter}}
	{15pt}
	{\bfseries \Large}{}

% Настройка вертикальных и горизонтальных отступов
\titlespacing*{\chapter}{0pt}{0pt}{8pt}
\titlespacing*{\section}{\parindent}{*4}{*4}

% Название для оглавления с отступом после него на 2,5 см
\AtBeginDocument{\renewcommand{\contentsname}{\begin{center} \vskip-2.5cm \Large{Содержание}\end{center}}}
\AtBeginDocument{\renewcommand{\bibname}{}}

% Настройка отображения главы в содержании
\titlecontents{chapter}% <section-type>
	[0pt]% <left>
	{}% <above-code>
	{\bfseries\chaptername~\thecontentslabel.~}% <numbered-entry-format>
	{}% <numberless-entry-format>
	{\bfseries\titlerule*[1pc]{.}\contentspage}% <filler-page-format>

% Настройка отображения раздела в содержании	
\titlecontents{section}% <section-type>
	[1.25cm]% <left>
	{}% <above-code>
	{\thecontentslabel~}% <numbered-entry-format>
	{}% <numberless-entry-format>
	{\titlerule*[1pc]{.}\contentspage}% <filler-page-format>
	
% Точка в номере параграфа и значок параграфа
%\renewcommand{\thesection}{\S~\arabic{chapter}.\arabic{section}.}

% Сброс нумерации формул, рисунков, таблиц
%\renewcommand{\theequation}{\arabic{equation}}
\usepackage{chngcntr}
\counterwithout{equation}{chapter}
\counterwithout{figure}{chapter}
\counterwithout{table}{chapter}

% Установка шрифтов
\usepackage{sectsty}
\chapterfont{\Large}
\sectionfont{\large\centering}
\paragraphfont{\normalsize}
\usepackage[font=small]{caption}

%%%
\newtheorem{teor}{Теорема}
\newtheorem{defin}{Определение}

% Точка вместо двоеточия в названиях рисунков и таблиц
\usepackage[labelsep=period]{caption}

% Цвет текста
\usepackage{xcolor}

% Clicable TOC
\usepackage{hyperref}
\hypersetup{
	colorlinks,
	citecolor=black,
	filecolor=black,
	linkcolor=black,
	urlcolor=black
}

% Список литературы, убрать лишнюю страницу
\usepackage{etoolbox}
\patchcmd{\thebibliography}{\chapter*{\bibname}}{\section*{\bibname}}{}{}

% This pack­age gives you easy ac­cess to the Lorem Ip­sum dummy text; an op­tion is avail­able to sep­a­rate the para­graphs of the dummy text into TEX-para­graphs.
\usepackage{lipsum}

\begin{document}
	\thispagestyle{empty}
	\begin{center}
		\textsc{Санкт-Петербургский государственный университет \\
			\textbf{Кафедра информационных систем}} \\
		\vspace{1cm}
		\Large{\textbf{ФИО полностью}} \\
		\vspace{1cm}
		\large{\textbf{Магистерская диссертация}} \\
		\vspace{2cm}
		\Large{\textbf{Название диссертации}} \\
		\normalsize{Направление 010402 \\
			Прикладная математика и информатика \\
			Математическое моделирование в задачах естествознания} \\
	\end{center}
	\vspace{2cm}
	\hspace{9cm} Научный руководитель, \\
	\hspace*{9cm} доктор физ.-мат. наук, \\
	\hspace*{9cm} профессор \\
	\hspace*{9cm} Матросов А. В.\\
	\begin{center}
		\vfill
		Санкт-Петербург \\
		2017
	\end{center}
	\newpage
	%%%%% CONTENTS %%%%%
	\tableofcontents
	\newpage
	
	\chapter*{Введение}
	\addcontentsline{toc}{chapter}{\bfseries Введение}
	\lipsum[1-5]
	
	\chapter*{Постановка задачи}
	\addcontentsline{toc}{chapter}{\bfseries Постановка задачи}
	\begin{figure}[h]
		\centering
		\includegraphics[width=\textwidth]{images/foobar.eps} 
		\caption{Image caption}		
		\label{fig:foobar}		
	\end{figure}
	\lipsum[6-12]
	
	\chapter*{Обзор литературы}
	\addcontentsline{toc}{chapter}{\bfseries Обзор литературы}
	\lipsum[13-17]
	
	\chapter{Название первой главы}
	\lipsum[18-19]
	\section{Первый параграф}
	\lipsum[19-24]
	\section{Второй параграф}	
	\lipsum[24-29]
	\section{Третий параграф}
	\lipsum[29-34]
	
	\chapter{Название второй главы}
	\lipsum[35-36]
	\section{Первый параграф}
	\lipsum[37-42]
	\section{Второй параграф}	
	\lipsum[42-47]
	\section{Третий параграф}
	\lipsum[47-52]
		
	\chapter{Название третьей главы}
	\lipsum[52-53]
	\section{Первый параграф}
	\begin{figure}[h!]
		\centering
		\subfigure[]{
			\includegraphics[width=0.48\textwidth]{images/foo.eps}
			\label{fig:foo}
		}
		\hspace{-2ex}
		\subfigure[]{
			\includegraphics[width=0.48\textwidth]{images/bar.eps}
			\label{fig:bar}
		} 
		\caption{Image caption: \subref{fig:foo} Foo; \subref{fig:bar} Bar}		
		\label{fig:double-foo-bar}		
	\end{figure}
	\lipsum[53-58]
	\section{Второй параграф}	
	\lipsum[58-63]
	\section{Третий параграф}
	\begin{table}[h!]
		\small
		\centering
		\begin{tabular}{|c|c|c|}
			\hline
			N & Foo & Bar
			\\ \hline
			%			1   & $\pm$ 0.44969469543403744 & $\pm$ 0.38010954324494359 \\ \hline
			%			2   & $\pm$ 0.44216359144626031 & $\pm$ 0.45870123259724908 \\ \hline
			%			3   & $\pm$ 0.44133731310780945 & $\pm$ 0.44056187347829946 \\ \hline
			%			4   & $\pm$ 0.44137288286167477 & $\pm$ 0.44001390378723744 \\ \hline
			5   & $\pm$ 0.44123418885718421 & $\pm$ 0.44570319394514344 \\ \hline
			%			10  & $\pm$ 0.44130320687361457 & $\pm$ 0.44128179701095466 \\ \hline
			%			20  & $\pm$ 0.44130147348281517 & $\pm$ 0.44130150495157090 \\ \hline
			30  & $\pm$ 0.44130125252097399 & $\pm$ 0.44130125857966327 \\ \hline
			%			40  & $\pm$ 0.44130119770410487 & $\pm$ 0.44130119991834346 \\ \hline
			50  & $\pm$ 0.44130117526630928 & $\pm$ 0.44130117622863519 \\ \hline
			%			60  & $\pm$ 0.44130116393968763 & $\pm$ 0.44130116440223840 \\ \hline
			%			70  & $\pm$ 0.44130115768329218 & $\pm$ 0.44130115792408247 \\ \hline
			80  & $\pm$ 0.44130115405877634 & $\pm$ 0.44130115419354609 \\ \hline
			%			90  & $\pm$ 0.44130115188889646 & $\pm$ 0.44130115196964533 \\ \hline
			100 & $\pm$ 0.44130115055301562 & $\pm$ 0.44130115060453695 \\ \hline
		\end{tabular}
		\caption{Table caption}
		\label{table:combined_3p-vx-vy-coeff}
	\end{table}
	\lipsum[63-68]
		
	\chapter*{Выводы}
	\addcontentsline{toc}{chapter}{\bfseries Выводы}
	\lipsum[69-73]
		
	\chapter*{Заключение}
	\addcontentsline{toc}{chapter}{\bfseries Заключение}
	\lipsum[74-78]
	
	\chapter*{Список литературы}
	\addcontentsline{toc}{chapter}{\bfseries Список литературы}	
	\begin{thebibliography}{9}
		\bibitem{bib:matrosov-1}
		Матросов~А.\:В.
		Вычислительная устойчивость алгоритма метода начальных функций
		// Вестник Санк-Петербургского университета. Серия 10. Прикладная математика. Информатика. Процессы управления, 2010. № 4. С. 30--39.	
		
		\bibitem{bib:matrosov-2}
		Матросов~А.\:В.
		Сходимость степенных рядов в методе начальных функций
		// Вестник Санкт-Петербургского университета. Серия 10. Прикладная математика. Информатика. Процессы управления, 2012. № 1. С. 41-51.	
		
		\bibitem{bib:matrosov-3}
		Matrosov~A.\:V.
		A numerical-analytical decomposition method in analyses of complex structures
		// 2014 International conference on computer technologies in physical and engineering applications (ICCTPEA)
		/ Editor: E.\:I. Veremey. St.-Petersburg: SPSU, 2014. P. 104--105.
	\end{thebibliography}

	\chapter*{Приложение A}
	\addcontentsline{toc}{chapter}{\bfseries Приложение A}
	
	\chapter*{Приложение B}
	\addcontentsline{toc}{chapter}{\bfseries Приложение B}
	
	\chapter*{Приложение C}
	\addcontentsline{toc}{chapter}{\bfseries Приложение C}
\end{document}